\documentclass{article}

\begin{document}

\title{SNEWS Operation Mode Description}         
\author{SNEWS Collaboration}
\date{\today}
\maketitle

\section{Introduction}      
The purpose of this document is to specify a template for operational
mode documents of the SuperNova Early Warning System (SNEWS).

It is the conviction of the participants that SNEWS should operate
under conditions that are:

\begin{itemize}
\item Documented and available at all times to the participants.
\item Approved by the participants.
\item Consistent with the objectives of SNEWS, those being:
\begin{itemize}
\item ``Prompt'':
\item ``Positive''
\item ``Private''
\end{itemize}
\end{itemize}

It is also implicit that SNEWS will develop in a series of managed
transitions between operational modes. For instance, new experiments
or new coincidence servers will be added or removed.  Each operational
mode will be identified by a number and the date when it came into
effect, and will specify in detail the participants, the coincidence
conditions, the alert classifications, and the procedures for action
in case of different alarm conditions.  This document outlines a
template for a operational mode document.

One caveat is that any details that would assist individuals in locating
and gaining access to the SNEWS servers, and information that could be
used to simulate SNEWS alerts, will be confined to an Appendix which
will not be widely distributed but is nonetheless available to certain
members of the participating experiments.


\section{Operational Mode Template}

\subsection{Participating Experiments}

This section will list all experiments that may send alerts to the
coincidence server(s).  It is understood that a temporary lack of
participation of one or more or these listed experiments for hours,
days, or weeks does not constitute a change of operational mode.  Any
other change in this list does constitute a change of Operational
Mode. However it is assumed that long lasting downtime periods
($\delta t>$1 day), previuosly scheduled for maintenance, calibration
or other items, must be officially notified to the SNEWS
Collaboration. The information will be posted into the downtime web
page.

$(please note: this choice can be discussed. Should the web page be public? 
Otherwise we can use e-mail notification to a restricted  subgroup inside the 
collaboration).$

 For example, the list could be:

\begin{itemize}
\item Super-K
\item LVD
\item SNO
\end{itemize}

\subsection{Client Computer Management}

Client computer management at each local site is a duty of each
participant experiment. No specific requirements are formally imposed
on the collaboration members. Basic recommendations will however
suggested here to improve the security of the whole system, since the
information passed via socket connection to the SNEWS server(s) are
present on the local machines.


\subsection{Server Computer Management}

The SNEWS server are managed according to memoranda
of understanding with national laboratory or similar sites;
this section will list the currently active coincidence
servers and provide pointers to the currently active memoranda
of understanding.  A significant MOU change, or permanent addition
or removal of a coincidence server from the network,  
constitutes a change of operational mode.

%\begin{itemize}
%\item account management / password management
%\item generic access guidelines
%\item physical security
%\item network security
%\item OS maintenance schedule
%\item Etc. etc.
%\end{itemize}

\subsection{Client-Server Communications}

This section will define individual experiment alarm categories 
($e.g.$ test, silver, gold) and will describe the 
communications protocol between client and server (encryption
protocol, access check requirements, etc).

\section{SNEWS Shift Work}

The SNEWS subgroup members will share SNEWS shift work on
a regular basis.  Each SNEWS member will be equipped with
communications equipment.
and is responsible 24/7 for response to SNEWS coincidences
(silver or gold), as well as some regular maintenance and monitoring jobs.
This section will outline the shifting mode and designate
the shiftmeister, and will list specific responsibilities of the 
SNEWS shifter.


\subsection{Coincidence Definition}

This section will define the coincidence conditions for different
SNEWS alert categories ($e.g.$ test, silver, gold).

\subsection{Alert Procedure}

This section will describe in detail the procedure to be followed by
SNEWS members and individual experiments for the case of each type of
coincidence alert.

\subsection{Communication to the Scientific Community}

This section will describe the means of communication
of the alert information to the astronomical community.

\subsection{Privacy Agreement}
This section will contain a pointer to the current SNEWS privacy
agreement, which defines guidelines for communication
of SNEWS-related information.

\end{document}
