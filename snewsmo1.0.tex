\documentclass{article}

\begin{document}

\title{SNEWS Operational Mode 1.0}         
\date{Effective \today}
\maketitle

This document provides specifications for the instance of
the class defined in the SNEWS Operational Mode Template document.


\section{Participating Experiments}

The participating experiments for this mode are:

\begin{itemize}
\item Super-K,
\item LVD,
\item SNO.
\end{itemize}

\section{Privacy Agreement}\label{privacy}

The SNEWS inter-experiment privacy agreement can be found at \\
{\tt http://cyclo.mit.edu/snnet/wg}.

\section{Client Computer Management}

Criteria of the privacy agreement (section~\ref{privacy}) regarding
access to SNEWS-related accounts on the client computer should be
strictly observed In addition, the following are recommended:

\begin{itemize}
\item Outside the main domain of the 
client computer, TCP/IP access to the client machine should be restricted.
\item Login to the client computer should permitted only to SNEWS subgroup 
members.
\item The TCP/IP address of the client computer should be known only to 
SNEWS subgroup members.
\end{itemize}

\section{Server Computer Management}

There is one server machine located at Brookhaven
National Laboratory, and the BNL SNEWS sysadmin is Brett Viren.
The details of the server management, which address all
security and maintenance issues, can be found
in the Memorandum of Understanding at\\
{\tt http://cyclo.mit.edu/snnet/bnl\_mou}.

\section{Client-Server Communications}\label{alarm}
\begin{itemize}

\item Each participating experiment may generate and send to the server
for different types of alarm datagrams: TEST, GOOD, DUBIOUS and RETRACTED.
{\it These designations may change}

\begin{itemize}

\item TEST: This flag indicates a datagram packet intended for test use
as well as for any high-rate test mode.  The alarm level
is set to {\it 0}.

\item DUBIOUS : This flag indicates an alarm
generated during scheduled operations (i.e. maintenance, calibration,
tests, etc.) or other known anomalous conditions. It is up to each
experiment to set this flag inside the packet when appropriate.  The
alarm level is set to {\it 1}.

\item GOOD: This flag indicates an alarm generated during
normal detection conditions.  The alarm level is set to {\it 2}.

\item RETRACTED: This flag indicates an alarm 
that has been retracted.  The alarm level is set to {\it -1}.

\end{itemize}
 
\item The details of the SNEWS datagram structure should
be known only to SNEWS subgroup members {\it but note some
working group members have worked with the code already} 

\item The client datagrams employ TCP protocol and 
are encrypted using OpenSSL.  
The server process listens for connections on a specified port,
and when a client initiates a connection, it employs several layers
of checks to validate the origin of the datagram:
\begin{enumerate}
\item The server process employs a {\tt tcp\_wrappers hosts\_access}
call to check the IP of the client machine against the lists in
{\it hosts.allow/deny}.  These lists allow only the IP addresses
of the client machines of the involved experiments.
\item
The client and server exchange certificates which have been
verified by the SNEWS Certificate Authority, and reject
connections if the check fails.
\item There is an additional check of specific client machine
IP addresses against a list of allowed addresses before data is exchanged.
\end{enumerate}
\end{itemize}

\section{SNEWS Shift}

SNEWS shifts will be coordinated by xxx, and will be run on
a week by week basis.  

%During each shift week, the shifter will
%be equipped with a cell phone and laptop with cellular
%capability  -> perhaps only email for first mo

\subsection{Alert Responsibilities}

Upon receipt of any coincidence message, either SILVER or GOLD, the
shifter must immediately log on to the server computer to check that
SNEWS code is running correctly, according to logs.  If everything
in order, he or she must follow
the alert procedures outlined in sections \ref{SILVER}-\ref{GOLD}.  ({\it If
something seems to be wrong, whatever the alert is, the consequent
actions for the shifter must be here defined.})

\subsection{Maintenance and Monitoring Responsibilities}

At daily intervals, the shifter must check the following:

\begin{itemize}
\item The gcserver process is running normally, and log output seems normal.
\item That there is network connectivity to all clients.
%\item Any other checks... pings etc.

\end{itemize}

The shifter must make an entry in the online logbook and record any
anomalous conditions, and, as appropriate notify the rest of the subgroup
of any problems.

% Backup shifter?


\section{Coincidence Definition}

The general coincidence definition implemented in the coincidence code
may generate either one two types of alert: GOLD and SILVER.\\
A GOLD alert is generated if {\it all} of
following conditions (1 through 4) are met:

\begin{enumerate}

\item There is a 2 or more -fold coincidence within 10 seconds,
involving at least two different experiments. 
(The time window refers to the maximum
separation of any of the alarms in the coincidence.)

\item At least two of the experiments involved
are at different laboratories.  This condition is automatically
satisfied for the current operational mode.

\item Two or more of the alarms in the coincidence
are flagged as GOOD.  It is the responsibility of each participant
experiment to flag the alarm sent to the SNEWS server(s)
appropriately. The specific criteria for a GOOD/DUBIOUS/TEST alarms
are locally defined by each experiment according to the rules defined
in section \ref{alarm}.
  
 \item For at least two of the experiments involved in the
 coincidence, the rate of good alarms for several past time intervals
 $\{T_i\}=\{$10 minutes, 1 hour, 10 hours, 1 day, 3 days, 1 week, 1
 month$\}$, must be consistent with the $\lambda_{\rm{max}}=$1/week
 requirement.\footnote{These intervals represent real time, not live
 time, since full live time information will not be available to the
 coincidence server.}  We define the precise condition as follows: if
 an experiment sent $\{n_i\}$ alarms in each of the last intervals
 $\{T_i\}$, then the Poisson probabilities $\mathcal{P}_i$ for $n_i$
 or more alarms in $T_i$,

$\mathcal{P}_i=\sum_{n=n_i}^{\infty}(\lambda_{\rm{max}} T_i)^{n}e^{-\lambda_{\rm{max}} T_i}/n!$,

for each interval $T_i$, must each be greater than $\mathcal{P}_{thr}=0.5$\%.
This corresponds to the condition that each $\{n_i\}$ must be be less
than $\{1,2,2,3,4,5,11\}$ for the preceding intervals $\{T_i\}$ for an
alarm to be GOLD.

\end{enumerate}

When the first criterion is satisfied, but at least one of the other
criteria are not satisfied, the generated alert is flagged as SILVER.
In this case the the alert has to be checked by the individual
experiment collaborations before any public announcement. No alert
will be sent to the community by SNEWS until (and if) there is an
upgrade to GOLD.

\section{Alert Procedure}
({\it This is a very important section. We are fixing the way to proceed
in the case of a coincidence. Please read carefully and add any 
comments/modifications/suggestions you think are useful. These are 
just the first ideas on. In some case  only the philosophy on what we need 
to do is specified with no technical solutions yet.}\\

\subsection{General Considerations and Procedures}

For both SILVER and GOLD cases, a message containing the following
information:

\begin{itemize}
\item UTC time of the coincidence,
\item all detectors involved in the coincidence, and
\item the types of alarms (GOOD, DUBIOUS) for each one involved
in the coincidence
\end{itemize}

will be automatically sent by the server to the SNEWS subgroup
members. 

({\it Should we post on-line the information in to a 
restricted WEB page on the server? May be useful in the case e-mail is not available. 
However the snnet web page should report the cronology of what has been going on.})

%It is up to each experiment to check
%its own data quality and forward to the SNEWS subgroup any information
%which may affect the alert type in upgrading or demoting it as well to
%resend the datagram to the server(s).  To avoid misunderstanding, the
%UTC time of a re-forwarded alarm must be preserved.
 
Note that the server process holds in memory one day's worth of
alarms.  

\subsection{GOLD Procedure}\label{GOLD}

For a GOLD alert, the alert message is sent directly to the 
{\tt snews-alert} mailing list, which includes addresses of all
who have signed up to received it. This includes astronomers,
and the Sky \& Telescope astro-alert list. 

\subsubsection{Individual Experiment Gold Procedures}

\begin{itemize}

\item SNO
\item Super-K
\item LVD

\end{itemize}


\noindent {\bf Demoting from GOLD:}\\ Although we hope
to avoid ever being in the situation where retraction of a GOLD alert
is necessary , any experiment may reflag from GOOD to RETRACTED its
own alarm after data checking.  The server will then automatically
reevaluate and reissue the alert based on alarms in its one-day
memory: the result may be still GOLD, demotion to SILVER, or no alert
at all.  For the latter case, a RETRACTED alert will be issued to the
same mailing list as for SILVER.

\subsection{SILVER Procedure}\label{SILVER}
For a SILVER alert, no automated alert message
is generated by the coincidence server. 
The procedure indicated by each experiment should be followed.\\

\subsubsection{Individual Experiment Silver Procedures}

\begin{itemize}

\item SNO
\item Super-K
\item LVD

\end{itemize}


\noindent \textbf{Upgrading from SILVER to GOLD}\\
A SILVER alert may be upgraded to GOLD if the requirements are
confirmed in a second step. Any individual experiment is expected to
resend its alarm datagram if the flag can be changed from DUBIOUS to
GOOD. All subgroup members must be informed at the same time by
e-mail. If the GOLD conditions are consequently fulfilled after the
new alarm dispatch, the GOLD procedure will automatically operate.

\subsection{SNEWS Data Exchange}

All server-client information
exchanged during a supposed SN alert will be saved
in the server log file. This information is available to subgroup
members at any time, and later to the experiment collaborations
by agreement of the SNEWS Advisory Board.

% info on web page for upgrade


\section{Communication to the Scientific Community}
There is no restriction on individual experiments making any announcement
based on individual observation in the case on absence of a SNEWS
alert, SILVER or GOLD, or preceding a SNEWS alert message.  
Any individual experiment may
publicly announce a supposed supernova signal following a dispatched 
SILVER alert which has not yet been upgraded to GOLD.  In this case 
the information that a previous SILVER alert from the
SNEWS server(s) has been received should be cited.
                                         
\end{document}
