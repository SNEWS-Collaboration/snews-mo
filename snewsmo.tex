\documentclass{article}

\begin{document}

\title{SNEWS Operation Mode Description}         
\author{SNEWS Collaboration}
\date{\today}
\maketitle

\section{Introduction}      
The purpose of this document is to specify a template for operational
mode documents of the SuperNova Early Warning System (SNEWS).

It is the conviction of the participants that SNEWS should operate
under conditions that are:

\begin{itemize}
\item Documented and available at all times to the participants.
\item Approved by the participants.
\item Consistent with the objectives of SNEWS, those being:
\begin{itemize}
\item ``Prompt'':
\item ``Positive''
\item ``Private''
\end{itemize}
\end{itemize}

It is also implicit that SNEWS will develop in a series of managed
transitions between operational modes. For instance, new experiments
or new coincidence servers will be added or removed.  Each operational
mode will be identified by a number and the date when it came into
effect, and will specify in detail the participants, the coincidence
conditions, the alert classifications, and the procedures for action
in case of different alarm conditions.  This document outlines a
template for a operational mode document.

One caveat is that any details that would assist individuals in locating
and gaining access to the SNEWS servers, and information that could be
used to simulate SNEWS alerts, will be confined to an Appendix which
will not be widely distributed but is nonetheless available to certain
members of the participating experiments.


\section{Operational Mode Template}

\subsection{Participating Experiments}

This section will list all experiments that may send alerts to the
coincidence server(s).  It is understood that a temporary lack of
participation of one or more or these listed experiments for hours,
days, or weeks does not constitute a change of operational mode.  Any
other change in this list does constitute a change of Operational
Mode.  For example, the list could be:

\begin{itemize}
\item S-K
\item LVD
\item SNO
\end{itemize}

\subsection{Client Computer Management}

This section will comprise a set of recommendations for
management of computers at the client sites, covering issues
of security and reliability.

(Note these will be recommendations rather than requirements).

\subsection{Server Computer Management}

The SNEWS server are managed according to memoranda
of understanding with national laboratory or similar sites;
this section will list the currently active coincidence
servers and provide pointers to the currently active memoranda
of understanding.  A significant MOU change, or permanent addition
or removal of a coincidence server from the network,  
constitutes a change of operational mode.

%\begin{itemize}
%\item account management / password management
%\item generic access guidelines
%\item physical security
%\item network security
%\item OS maintenance schedule
%\item Etc. etc.
%\end{itemize}

\subsection{Client-Server Communications}

This section will define individual experiment alarm categories 
($e.g.$ test, silver, gold) and will describe the 
communications protocol between client and server (encryption
protocol, access check requirements, etc).

\subsection{Coincidence Definition}

This section will define the coincidence conditions for different
SNEWS alert categories ($e.g.$ test, silver, gold).

\subsection{Alert Procedure}

This section will describe in detail the procedure to be followed by
SNEWS members and individual experiments for the case of each type of
coincidence alert.

\subsection{Communication to the Astronomical Community}

This section will describe the means of communication
of the alert information to the astronomical community.

\subsection{Privacy Agreement}
This section will contain a pointer to the current SNEWS privacy
agreement, which defines guidelines for communication
of SNEWS-related information.

\end{document}
