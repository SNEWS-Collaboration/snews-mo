\documentclass{article}

\begin{document}

\title{need a good title}         
\author{SNEWS Collaboration}
\date{\today}
\maketitle

\section{Objective}      
The object of this document is to completely specify the operational mode of the SuperNova Early Warning System (SNEWS).  

It is the conviction of the participants that SNEWS should operate under conditions that are
\begin{itemize}
\item Documented and available at all times to the participants
\item Approved by the participants
\item Consistent with the objectives of SNEWS, those being
\begin{itemize}
\item Prompt
\item the other P's including a statement about false alarm rates
\end{itemize}
\end{itemize}

One caveat is that details that would assist individuals in locating
and gaining access to the SNEWS servers, and information that could be
used to simulate SNEWS alerts, will be confined to an Appendix which
will not be widely distributed but is nonetheless available to certain
members of the participating experiments.

It is also implicit that SNEWS will develop in a series of managed
transitions between operational modes. Each operational mode will be
identified by a number and the date when it came into effect.
More�.

\section{Operational Mode \#1 effective 1/11/2003}

\subsection{The participants}
	(Here are listed all experiments that may presently send
alerts to the coincidence server(s).  It is understood that a
temporary lack of participation of one or more or these listed
experiments for hours, days, or weeks does not constitute a
change of operational mode.  Any other change in this list
does constitute a change of Operational Mode.)
\begin{itemize}
\item S-K
\item LVD
\item SNO
\end{itemize}

\subsection{Client Computer Management}


\subsection{Server Computer Management}
\begin{itemize}
\item account management / password management
\item generic access guidelines
\item physical security
\item network security
\item OS maintenance schedule
\item Etc. etc.
\end{itemize}

\subsection{Client-Server Communications}
\begin{itemize}
\item definition of test, silver, and gold (experiment) alerts
\item generic description of security measures / encryption 
\end{itemize}

\subsection{Coincidence Definition}

A ``gold'' alert is generated if \textit{all} of
following conditions are met:

\begin{enumerate}

\item There is a 2 or more -fold coincidence within 10 seconds,
involving at least
two different experiments.  (The time window refers to the maximum
separation of any of the alarms in the coincidence.)

\item At least two of the experiments involved
are at different laboratories.

\item Two or more of the alarms in the coincidence
are flagged as ``good''.  (Individual experiments flag
as ``good'' if running conditions are normal, e.g. no
calibrations, etc.  The specific criteria for a good alarm are defined
by each experimental collaboration).

\item For all of the experiments involved in the coincidence,
the rate of good alarms for several past time intervals $\{T_i\}=\{$10 minutes, 1
hour, 10 hours, 1 day, 3 days, 1 week, 1 month$\}$, must be consistent with
the $\lambda_{\rm{max}}=$1/week requirement.\footnote{These intervals
represent real time, not live time, since full live time information
will not be available to the coincidence server.}
We define the precise condition 
as follows:  
if an experiment sent $\{n_i\}$ alarms in 
each of the last intervals $\{T_i\}$,
then the Poisson probabilities $\mathcal{P}_i$ for $n_i$ or more
alarms in $T_i$,

$\mathcal{P}_i=\sum_{n=n_i}^{\infty}(\lambda_{\rm{max}} T_i)^{n}e^{-\lambda_{\rm{max}} T_i}/n!$,

for each interval $T_i$, must each be greater than $\mathcal{P}_{thr}=0.5$\%.
This corresponds to the condition that each $\{n_i\}$ must be be less
than $\{1,2,2,3,4,5,11\}$ for the preceding intervals $\{T_i\}$ for an
alarm to be ``gold''.

\end{enumerate}

A ``silver'' alert, to be checked by the experimenters,
satisfies the first criterion but not all of the others.


\subsection{Annunciation}
\subsubsection{Silver Procedure (including elevation to Gold)}

\subsubsection{Gold Procedure}


\subsection{Server - Astro-Community Communications}


\end{document}
