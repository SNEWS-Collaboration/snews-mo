\documentclass{article}

\begin{document}

\title{SNEWS Operational Mode \#1 effective 1/11/2003}         
\author{SNEWS Collaboration}
\date{\today}
\maketitle

\section{Participating Experiments}

\begin{itemize}
\item Super-K
\item LVD
\item SNO
\end{itemize}

\section{Client Computer Management}

The following are recommended:

\begin{itemize}
\item TCP/IP access should be restricted outside the main domain of the client Computer
\item Login should permitted only to SNEWS subgroup members.
\item The TCP/IP address of client computer should be known only to SNEWS subgroup members.
\item The full datagram structure should be known only to SNEWS subgroup members.
\item .. anything else..
\end{itemize}
Most part of the preceeding recommendations are already effective.

\section{Server Computer Management}

In this mode of operation, there is one server located at Brookhaven
National Laboratory, and the SNEWS sysadmin there is Brett Viren.
The details of the server management, addressing all
security and maintenance issues, can be found at
in the Memorandum of Understanding at
{\it http://cyclo.mit.edu/snnet/bnl\_mou}.

\section{Client-Server Communications}
\begin{itemize}
\item definition of test, silver, and gold (experiment) alerts
\item The client datagrams employ TCP protocol and 
are encrypted using OpenSSL.  
The server listens for connections on a specified port,
and when a client initiates a connection, several layers
of checks are employed:
\begin{enumerate}
\item The server employs a {\it tcp\_wrappers hosts\_access}
call to check the IP of the client machine against the lists in
{\it hosts.allow/deny}.  These lists allow only the IP addresses
of the client machines of the involved experiments.
\item
The client and server exchange certificates which have been
verified by the SNEWS Certificate Authority, and reject
connections if the check fails.
\item There is an additional check of specific client machine
IP addresses before data is exchanged.
\end{enumerate}
\end{itemize}

\section{Coincidence Definition}

The general Coincidence Definition implemented in the coincidence code
may generate two types of alert:``gold'' and ``silver''.\\
A ``gold'' alert is generated if {\it all} of
following conditions (1 through 4) are met:

\begin{enumerate}

\item There is a 2 or more -fold coincidence within 10 seconds,
involving at least two different experiments. 
(The time window refers to the maximum
separation of any of the alarms in the coincidence.)

\item At least two of the experiments involved
are at different laboratories.

\item Two or more of the alarms in the coincidence
are flagged as ``good''.  The SNEWS collaboration states
that flagging the alarms as GOOD or BAD (eventually TEST) should be effective 
for all detectors when in the operational  mode. 
Generally a ``good'' flag means a normal running condition, excluding 
calibrations, tests, and all purposes maintenance.  
The specific criteria for a GOOD/BAD/TEST alarm are locally defined by
each experiment. 
It's up to each participant to flag properly the alarm sent to the SNEWS 
server(s).  
 
\item For at least two of the experiments involved in the coincidence,
the rate of good alarms for several past time intervals $\{T_i\}=\{$10 minutes, 1
hour, 10 hours, 1 day, 3 days, 1 week, 1 month$\}$, must be consistent with
the $\lambda_{\rm{max}}=$1/week requirement.\footnote{These intervals
represent real time, not live time, since full live time information
will not be available to the coincidence server.}
We define the precise condition 
as follows:  
if an experiment sent $\{n_i\}$ alarms in 
each of the last intervals $\{T_i\}$,
then the Poisson probabilities $\mathcal{P}_i$ for $n_i$ or more
alarms in $T_i$,

$\mathcal{P}_i=\sum_{n=n_i}^{\infty}(\lambda_{\rm{max}} T_i)^{n}e^{-\lambda_{\rm{max}} T_i}/n!$,

for each interval $T_i$, must each be greater than $\mathcal{P}_{thr}=0.5$\%.
This corresponds to the condition that each $\{n_i\}$ must be be less
than $\{1,2,2,3,4,5,11\}$ for the preceding intervals $\{T_i\}$ for an
alarm to be ``gold''.

\end{enumerate}

When at least the first criterion is satisfied the generated alert is
flagged as ``silver''. In this case the consistency and the reliability 
of the alert has to be checked by the experimenters inside the collaboration before any public announcement. No automatic alert has to be sent.

\section{Alert Procedure}
({\it This is a very important section. We are fixing the way to procede
in the case of a coincidence. Please read carefully and add any 
comments/modifications/suggestions you think are useful. These are 
just the first ideas on. In some case  only the philosphy on what we need 
to do is specified with no technical solutions yet.}\\

Whichever the automated generated alert is -GOLD or SILVER- the general 
approved procedure here enclosed will be strictly applied. 
In the same context any public 
release both of the Collaboration either of each Experiment should respect
the general policies here defined.\\ 
In order to have the maximum knowledge on the alert structure the participant experiments 
agree that all relevant parameters  {\it -UTC time of the coincidence, 
involved detectors, type of alarms-} will be automatically forwarded by the server(s) to the 
SNEWS subgroup members at the time of the dispatched alert. They also agree that the UTC
time of the coincidence and the number of detectors which generate it may be used either for any
scientific purpose or for any outreach communication. 
In this case the SNEWS origin of these informations should be always clearly cited.

({\it Should we post on-line the information in to a 
restricted WEB page on the server? May be useful in the case e-mail is not available. 
However the snnet web page should report the cronology of what has been going on.})

Each experiment provides a list of e-mails/names and fix the specific notification procedure. 
It is up to each experiment members to check its own data quality and forward
to the SNEWS subgroup any information which may affect the alert type 
in upgrading or demoting it as well to resend the datagram to the server(s).
To avoid misunderstanding the UTC time of a re-forwarded alarm must be 
preserved.
 
Specific procedures for GOLD and SILVER alert and consequent actions 
are therefore discussed.

({\it Question: what's the current time limit to regenerate the coincidence?
How long do we wait? We need to define this to avoid retraction after long 
time. If another server will be on-line we have to fix some priority between 
them, ore we can have a double notification.
})

\subsection{Gold Procedure}
In the case of a GOLD alert all SNEWS collaboration members must be informed
on-line at the time of the automated alert is sent to the Astro-Community. 
Any outcoming, private or public, comunication of each Experiment following 
this moment should always 
mention the presence of the GOLD alert, possibly with all public informations
therein contained.\\
{\bf Demoting from GOLD to SILVER}\\
Any Experiment can individually reflag from GOOD to BAD its own alarm after
data checking and must procede as expected by the above general policies.
Doing this, the possibility of an automated retraction procedure for a GOLD 
alert is taken into account and it must be effective within a reasonnable
time to be processed by the server(s). 
In this case all SNEWS collaboration members as well as 
the Astro-Community must be immediately notified in the same way used to
send the GOLD one alert. The presence of a lower significance alert should 
still be included.({\it Does everyone agree on this point?})

\subsection{Silver Procedure}
No automated alert is generated by the coincidence server(s). 
Only the  SNEWS subgroup members are immediately notified 
with all informations needed for the prompt check of the data in order to 
confirm or reject it definitely.\\
\textbf{Upgrading from SILVER to GOLD}\\
A SILVER alert may be upgraded to GOLD if the requirements are confirmed in a second step. Any
individual Experiment is expected to resend its alarm datagram if the flag must be  changed from 
BAD to GOOD. All subgroup members must be informed at the same time
by e-mail. If the GOLD conditions are consequently fulfilled after the new alarm dispatch,  
the GOLD procedure will automatically operate.

\subsection{Final Agreements}

The SNEWS collaboration agrees that all server(s)-clients informations exchange 
during a supposed SN alert will be saved both on the wep page and the server(s) 
log file. All these informations will be available for all subgroup members
at time of the event and after. 
They will be released on request to the SNEWS collaboration members 24 (?) hours after the event.
({\it 24 hours may be a too long time?})
There's no restriction for any subgroup member to forward the informations
above inside the SNEWS collaboration  24 (?) hours after the event.


\section{Communication to the Scientific Community}
There is no restriction on invidual experiments announcing
something based on individual observation in the case on absence
of a SNEWS alert, silver or gold, or preceeding it.
The SNEWS Collaboration agrees also on individual public announcement of a 
supposed supernova signal following in time a dispatched silver alert
which has been not  yet upgraded to gold. If it is the case the SNEWS 
collaboration requires that the announcing experiment must clearly include the 
information that a previous ``silver'' alert from the SNEWS server(s) has been 
received.
                                         
\subsection{Privacy Agreement}
                                     

\end{document}
