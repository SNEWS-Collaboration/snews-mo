\documentclass{article}

\begin{document}

\title{SNEWS Operational Mode \#1 effective 1/11/2003}         
\author{SNEWS Collaboration}
\date{\today}
\maketitle

\section{Participating Experiments}

\begin{itemize}
\item Super-K
\item LVD
\item SNO
\end{itemize}

\section{Client Computer Management}

The following are recommended:

\begin{itemize}
\item TCP/IP access should be restricted outside the main domain of the client Computer
\item Login should permitted only to SNEWS subgroup members.
\item The TCP/IP address of client computer should be known only to SNEWS subgroup members.
\item The full datagram structure should be known only to SNEWS subgroup members.
\item .. anything else..
\end{itemize}
Most part of the preceeding recommendations are already effective.

\section{Server Computer Management}

In this mode of operation, there is one server located at Brookhaven
National Laboratory, and the SNEWS sysadmin there is Brett Viren.
The details of the server management, addressing all
security and maintenance issues, can be found at
in the Memorandum of Understanding at
\texttt{http://cyclo.mit.edu/snnet/bnl_mou}.

\section{Client-Server Communications}
\begin{itemize}
\item definition of test, silver, and gold (experiment) alerts
\item The client datagrams employ TCP protocol and 
are encrypted using OpenSSL.  
The server listens for connections on a specified port,
and when a client initiates a connection, several layers
of checks are employed:
\begin{enumerate}
\item The server employs a \texttt{tcp\_wrappers hosts\_access}
call to check the IP of the client machine against the lists in
\texttt{hosts.allow/deny}.  These lists allow only the IP addresses
of the client machines of the involved experiments.
\item
The client and server exchange certificates which have been
verified by the SNEWS Certificate Authority, and reject
connections if the check fails.
\item There is an additional check of specific client machine
IP addresses before data is exchanged.
\end{enumerate}
\end{itemize}

\section{Coincidence Definition}

The general Coincidence Definition implemented in the coincidence code
may generate two types of alert:``gold'' and ``silver''.\\
A ``gold'' alert is generated if \textit{all} of
following conditions (1 through 4) are met:

\begin{enumerate}

\item There is a 2 or more -fold coincidence within 10 seconds,
involving at least two different experiments. 
(The time window refers to the maximum
separation of any of the alarms in the coincidence.)

\item At least two of the experiments involved
are at different laboratories.

\item Two or more of the alarms in the coincidence
are flagged as ``good''.  The SNEWS collaboration states
that flagging the alarms as GOOD or BAD (eventually TEST) should be effective 
for all detectors when in the operational  mode. 
Generally a ``good'' flag means a normal running condition, excluding 
calibrations, tests, and all purposes maintenance.  
The specific criteria for a GOOD/BAD/TEST alarm are locally defined by
each experiment. 
It's up to each participant to flag properly the alarm sent to the SNEWS 
server(s).  
 
\item For at least two of the experiments involved in the coincidence,
the rate of good alarms for several past time intervals $\{T_i\}=\{$10 minutes, 1
hour, 10 hours, 1 day, 3 days, 1 week, 1 month$\}$, must be consistent with
the $\lambda_{\rm{max}}=$1/week requirement.\footnote{These intervals
represent real time, not live time, since full live time information
will not be available to the coincidence server.}
We define the precise condition 
as follows:  
if an experiment sent $\{n_i\}$ alarms in 
each of the last intervals $\{T_i\}$,
then the Poisson probabilities $\mathcal{P}_i$ for $n_i$ or more
alarms in $T_i$,

$\mathcal{P}_i=\sum_{n=n_i}^{\infty}(\lambda_{\rm{max}} T_i)^{n}e^{-\lambda_{\rm{max}} T_i}/n!$,

for each interval $T_i$, must each be greater than $\mathcal{P}_{thr}=0.5$\%.
This corresponds to the condition that each $\{n_i\}$ must be be less
than $\{1,2,2,3,4,5,11\}$ for the preceding intervals $\{T_i\}$ for an
alarm to be ``gold''.

\end{enumerate}

When at least the first criterion is satisfied the generated alert is
flagged as ``silver''. In this case the consistency and the reliability 
of the alert has to be checked by the experimenters inside the collaboration before any public announcement. No automatic alert has to be sent.


\section{Alert Procedure}
Both in case of GOLD either SILVER alert a standard approved procedure must 
be followed before any public release. 

\subsection{Gold Procedure}
In the case of a GOLD alert all SNEWS collaboration member must be informed
on-line when the automated alert has been sent to the Astro-Community. It is
up to each experiment member to check the own data quality and reply to the 
server with a demoted ``bad'' alert if it is the case.
This includes therefore the possibility of a retraction procedure for a gold 
alert that must be effective within a reasonnable time.

\subsection{Silver Procedure}
No automated alert should be sent by the coincidence server(s). However,
since the SNEWS goal in this case will be the most quick check of the detector signals to confirm or reject it definitely,
a subset of people within the whole collaboration should be promptly notified
as in the case of ``gold'' alert. 
Each experiment provides a list of e-mails/names and fix the
specific notification procedure. SNEWSmo provides pointers to individual procedures. 

\textbf{Upgrading Silver to Gold}
A silver alert may be upgraded to gold as follows:  an
individual experiment can resend its alarm datagram with ``bad''
changed to ``good'', which may automatically cause an alert message to
astronomers to be sent if gold conditions are fulfilled.  

\section{Communication to the Scientific Community}
There is no restriction on invidual experiments announcing
something based on individual observation in the case on absence
of a SNEWS alert, silver or gold, or preceeding it.
The SNEWS Collaboration agrees also on individual public announcement of a 
supposed supernova signal following in time a dispatched silver alert
which has been not  yet upgraded to gold. If it is the case the SNEWS 
collaboration requires that the announcing experiment must clearly include the 
information that a previous ``silver'' alert from the SNEWS server(s) has been 
received.
                                         
\subsection{Privacy Agreement}
                                     

\end{document}
