\documentclass{article}

\begin{document}

\title{SNEWS Operation Mode Description: Appendix}         
\author{SNEWS Collaboration}
\date{\today}
\maketitle

This document, which should
be easily accessible to the subgroup only,
contains operating mode details which are not intended
for wide distribution.  

\section{Datagram Packet Definitions}      

The C structures of the alarm datagram packets of the three types 
(PING, ALARM, RETRACTION) are shown below.\\


\noindent
\underline{\textbf{PING packet:}}

\begin{verbatim}
struct check_packet {           /* packet_id = 3 */
        u_short packet_id;      /* What type of packet? */  
        u_short experiment;     /* Which experiment? */
        LONG   datetime[3];     /* Time word : ddmmyy,hhmmss,ns */
        ULONG   sequence;       /* Counter to allow send/receive checks */
};
\end{verbatim}

\noindent
\underline{\textbf{ALARM packet:}}

\begin{verbatim}
struct alarm_packet {   /* packet_id = 1 */
  u_short packet_id;    /* What type of packet? */  
  u_short experiment;   /* Which experiment? */
  u_short level;        /* Alarm level */
  u_short dummy;
  ULONG   sequence;     /* Counter to allow send/receive checks */
  LONG   datetime[3];   /* Time word : ddmmyy,hhmmss,ns */
  ULONG   signif;       /* The experiment's signif. estimate *10^6 */
};

\end{verbatim}

\noindent
\underline{\textbf{RETRACTION packet:}}

\begin{verbatim}
struct retract_packet { /* packet_id = 2 */ 
  u_short packet_id;    /* What type of packet? */   
  u_short experiment;   /* Which experiment? */ 
  u_short level;        /* necessary for high-rate test */ 
  LONG  starttime[3];   /* datetime to start retracting packets */ 
  LONG  endtime[3];     /* datetime to end retracting packets */ 
  ULONG   sequence;     /* Counter to allow send/receive checks */ 
}; 

\end{verbatim}
Note that not all fields of the packets are used by the server.

\section{Call Lists}      

\end{document}
